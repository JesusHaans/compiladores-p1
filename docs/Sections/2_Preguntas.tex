\section{Preguntas}

\subsection{Menciona 4 tipos de malware y explica su funcionamiento.}

\begin{enumerate}
    \item Virus: Se adjunta a programas/archivos legítimos, se propaga al ejecutarse, y puede corromper o eliminar datos.
    \item Gusano: Se auto-replica y se propaga a través de redes, explotando vulnerabilidades sin necesidad de intervención del usuario.
    \item Troyano: Se disfraza de software legítimo para engañar a los usuarios y realizar actividades maliciosas.
    \item Ransomware: Cifra archivos o bloquea sistemas y exige un rescate para restaurar el acceso.
\end{enumerate}

\subsection{¿Cómo funciona el malware llamado Pegasus?}

Pegasus es un spyware que, una vez instalado en el teléfono, es capaz de activar el micrófono y la cámara del terminal desde un panel de control remoto. Además, hace las veces de geolocalizador lo que le permite conocer con exactitud dónde se encuentra el propietario. 

\subsection{¿Cuál fue el alcance del virus WannaCry?}

El ransomware conocido como WannaCry conmocionó al mundo con un ataque fulgurante: en un solo día infectó más de 230 000 PC con Windows en 150 países, muchos de ellos pertenecientes a agencias gubernamentales y hospitales.

\subsection{¿Cuál sería las consecuencias si un malware con el mismo alcance del de Morris nos infectara hoy en día?}

Teniendo en cuenta que el numero de usuarios roza los 5 millones teniendo en cuenta que Morris tuvo un alcance de un 10\% estamos hablando de 500,000 usuarios infectados suponiendo en el mejor de los casos de puros usuarios promedio sin contar empresas o especialistas en un enorme alcance el que podría tener un virus de esa magnitud a eso sumarle lo demás que tenga cargado el gusano se vuelve algo peligroso ya que si es una forma de robar información es  demasiada la cual podrían hacer mal uso de ella como ya se vio en practicas pasadas y si se desea tener bots para ataques de denegación de servicios yo considero es bastante fácil poder tirar cualquier sitio con un dominio tan grande de bots para ese fin. 

\subsection{¿Cómo previenes el malware?}

Para un usuario es muy importante contar con algún antivirus ya que esto podría ser la primera linea de fuego y ayudarlos a evitar alguna infección de malware. Además  de siempre verificar el origen de los archivos, no navegar por sitios que puedan ser inseguros, no descargar archivos de personas que no conozcas.

\subsection{Investiga sobre CovidLock. ¿Cómo funciona? ¿Cuál fue su alcance?}

CovidLock, una aplicación para Android que promete mantener a los usuarios informados sobre los casos de Covid-19 en todo el mundo. Sin embargo, en realidad, es un malware. Cuando se descarga y ejecuta, elimina el código de bloqueo del dispositivo y lo reemplaza con uno definido por el hacker. Luego, exige un rescate de 100 dólares en Bitcoins para recuperar el control del dispositivo.

Si el rescate no se paga en 48 horas, amenazan con borrar todo el contenido del dispositivo y hacer pública la información personal que contenga. Los investigadores que descubrieron este malware advierten que se trata de un ransomware sin precedentes para Android.

La lección aquí es instalar aplicaciones solo desde la tienda oficial de Google y fuentes verificadas para evitar sorpresas desagradables. La descarga de archivos de fuentes no confiables puede resultar costosa en términos de seguridad y privacidad.

\subsection{Menciona un caso donde un keylogger fuera usado lícitamente, no mencionado previamente.}

Me parece una total invasión a la privacidad no debería poderse recurrir lícitamente mas que en los casos ya mencionados en la practica.

\subsection{Menciona un caso donde un keylogger creo conmoción por culpa de ciberdelincuentes, no mencionado previamente.}

Un estudiante irá preso por vender un keylogger que infectó a 16.000 personas.

Zachary Shames, un estudiante universitario de 21 años, se ha declarado culpable de desarrollar y comercializar un software malicioso conocido como keylogger. Este software fue vendido a aproximadamente 3 mil personas, quienes lo utilizaron para infectar más de 16.000 computadoras con el fin de robar información sensible, como contraseñas y credenciales bancarias.

Shames comenzó a desarrollar el keylogger mientras aún estaba en la escuela secundaria y continuó modificándolo mientras estaba en la universidad. Vendió el software ilegalmente desde su dormitorio universitario hasta que fue arrestado en marzo de 2016 durante un allanamiento policial.

El Departamento de Justicia de Estados Unidos ha presentado cargos contra Shames por ayuda e incitación a intrusiones informáticas. Se enfrenta a una sentencia máxima de 10 años de prisión y será sentenciado el 16 de junio.

El keylogger se vendía por 25 dólares a través de PayPal, pero se encontraba disponible por menos en el mercado underground. Además de robar credenciales y registrar pulsaciones de teclado, el software enviaba la información a un sitio web llamado limitlessproducts.org, propiedad de Shames.

Para evadir la detección de antivirus, Shames cifraba el software y proporcionaba instrucciones para su uso y cifrado en videos en YouTube. Sin embargo, Shames dejó rastros al registrar el dominio del sitio web con su información personal, lo que facilitó al FBI su identificación y arresto.

\subsection{¿Crees que es ético el uso de keylogger en el hogar? (Espiando hijos, hermanos, padres o parejas)}

no me parece etico ya que invades la privacidad de tu familia y eso no es correcto lo mejor es tener charlas tipo capacitaciones de ciberceguridad con cada miembro para evitar cualquier peligro que se pudiera dar

\subsection{¿Crees ético el uso de keylogger en cualquier espacio, ya sea lícita o ilícitamente?}

solo en el espacio licito bajo amenasas de estado como para que una organizacion gubernamental apruebe ese uso pero de ahi en fuera no me parece correcto el uso de este malware.




