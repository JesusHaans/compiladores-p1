\singlespacing
\section{Introducción}

En esta practica veremos que son los keyloggers, son herramientas diseñadas para registrar todo lo que se escribe en un teclado de computadora o dispositivo móvil. Se utilizan para monitorear de manera encubierta la actividad del usuario, con fines tanto legítimos como malintencionados. Los desarrolladores de software pueden usarlos para recibir comentarios, pero también pueden ser empleados por delincuentes para robar datos personales.

El término "keylogger" se refiere tanto al acto de registrar pulsaciones de teclas como a las herramientas que lo permiten. Estas herramientas pueden encontrarse en diversos entornos, desde dispositivos personales hasta servidores empresariales, y pueden ser utilizadas por ciberdelincuentes, parejas sospechosas o incluso empresas.

\subsection{Funcionamiento de los keyloggers}

Los keyloggers registran cada tecla presionada, a menudo sin el conocimiento del usuario, incluyendo la duración, hora, velocidad y nombre de la tecla. Esta información, que puede incluir datos sensibles como credenciales bancarias y mensajes personales, es interceptada y almacenada.

\subsection{Tipos de keyloggers}

\textbf{Keyloggers de software:}

\begin{enumerate}
    \item Keyloggers en API: Interceptan señales de teclado a nivel de aplicación.
    \item Keyloggers de formularios: Registran datos ingresados en formularios web.
    \item Keyloggers de kernel: Acceden al núcleo del sistema para obtener permisos avanzados.
\end{enumerate}

\textbf{Keyloggers de hardware:}

\begin{enumerate}
    \item De teclado: Integrados en el cable o el teclado mismo.
    \item De cámara oculta: Ubicados en lugares públicos para registrar visualmente las pulsaciones.
    \item En unidades USB: Dispositivos físicos que instalan malware al conectarse.
\end{enumerate}